\documentclass{beamer}

%\usepackage{listings}
%\usepackage[francais]{babel}
\usepackage[T1]{fontenc}
\usepackage[utf8]{inputenc}
%\usepackage{MyriadPro}
%\usepackage{cabin}
\usepackage{graphicx}
\usepackage{array}
\usepackage{tikz}
\usetikzlibrary{positioning, backgrounds, shapes, chains, arrows, decorations.pathmorphing}

\usepackage{forest}

\usepackage{amsmath,amsthm,amssymb}  
\usepackage{stmaryrd}
%\usepackage{mdsymbol}
\usepackage{MnSymbol}
\usepackage{xcolor}
\usepackage{verbatim}
\usepackage{array}
\usepackage{gb4e}
%\usepackage{csquotes}

\usepackage{cancel}



\usepackage[absolute,overlay]{textpos}
%\usepackage[texcoord,
%grid,gridcolor=red!10,subgridcolor=green!10,gridunit=pt]
%{eso-pic}



%\useoutertheme{infolines}
\usetheme{focus}

\newcommand{\hidden}[1]{}

%colors
\definecolor{darkgreen}{rgb}{0,0.5,0}
\usebeamercolor{block title}
\definecolor{beamerblue}{named}{fg}
\usebeamercolor{alert block title}
\definecolor{beamealert}{named}{fg}

\renewcommand{\colon}{\!:\!}


\newcommand\paraitem{%
 \quad
 \makebox[\labelwidth][r]{%
 \makelabel{%
 \usebeamertemplate{itemize \beameritemnestingprefix item}}}\hskip\labelsep}

\newcommand{\mmid}{\mathbin{{\mid}{\mid}}}


%tikz stuff

\begin{document}

\title{Semantic expressive capacity with bounded memory} 
\author{Antoine Venant}
\institute{Saarland University}
\date{\today \\ \vspace{1cm} \\ joint work with Alexander Koller}
\maketitle

\begin{frame}\frametitle{Semantic interpretation}
    Linguistic expression $\Rightarrow$ (formal) meaning representation.\\ Representations can be logical formulae, or graphs (AMR [Banarescu \& all 2013], MRS [Copestake \& all 2005]).
    \begin{center}
      \begin{tikzpicture}[g/.style = {draw, circle, black}]     
        \node[rectangle] (s) at (0,0) {\small Morgane wants to sleep};
        \node (t) at (2,0) {};
        \begin{scope}[xshift = 4cm, yshift=0.35cm]
          \node[g] (v) at (1.25,1.25) {};
          \path (v) edge[loop above] node[above] {\small $\mathsf{want}$} (v);
          \node[g] (m) at (0.5,0) {};
          \path (m) edge[loop left] node[above] {\small $\mathsf{Morgane}$} (m);
          \node[g] (d) at (2,0) {};
          \path (d) edge[loop right] node[above] {\small $\mathsf{sleep}$} (d);
          \path (v) edge[->] node[left, midway] (ref) {\small $\mathsf{a}_0$} (m);
          \path (v) edge[->] node[right, midway] {\small $\mathsf{a}_1$} (d);
          \path (d) edge[->] node[above] (a0) {\small $\mathsf{a}_0$} (m);
        \end{scope}
        \node[xshift=-1.75cm] (ref2) at (ref) {};
        \path (s.east) edge[double, ->] (ref2);
        
        \node[yshift = -2cm] (ref3) at (ref2) {};
        \node[yshift=-1.5cm, xshift=0.75cm] (l) at (a0) {\small $\mathsf{Morgane}(x) \wedge \mathsf{want}(e, x, \mathsf{sleep}(e', x))$};
        \path (s.east) edge[double, ->] (ref3);
      \end{tikzpicture}
    \end{center}

    \begin{itemize} 
    \item Consensual approach: semantic interpretation is a \emph{compositional} process, guided by syntax.
  \end{itemize}
  
\end{frame}

\begin{frame}{Principle of compositionality}
  \begin{block}{Statement}
    ``The meaning of a complex expression is a \textbf{function} of the meaning of its parts and the \textbf{syntactic rule} that combines them.''
  \end{block}

  \begin{block}{Requires:}
    \begin{itemize}
    \item A syntax tree, along which semantic construction is performed in a bottom-up fashion.
    \item Operators for semantic composition (semantic algebra).
    \end{itemize}
  \end{block}

  \begin{itemize}
  \item \alert{Which semantic interpretation functions can we express \emph{compositionally} using specific classes of syntax trees \textbf{and} semantic operators?}
  \end{itemize}
\end{frame}

\begin{frame}{Semantic algebra}
  \begin{block}{(Essentially) one job:}
    %Main job:
    combine predicates with their arguments.
  \end{block}
  
  \begin{center}
    \begin{tikzpicture}[g/.style={draw, circle, black}]
      \node[g] (m) at (-1,0) {};
      \path (m) edge[loop left] node[above] {\small $\mathsf{Morgane}$} (m);
      
      \node[g] (d) at (1,0) {};
      \node[] (unk) at (1, -1) {$\langle \mathsf{s} \rangle$};
      \path (d) edge[loop right] node[above] {\small $\mathsf{sleep}$} (d);
      \path (d) edge[->] node[left, midway] (ref) {\small $\mathsf{a}_0$} (unk);
      \path[draw, dashed, red, thick, ->] (m) ..controls (-0.5,-1.5) and (0.55,-1).. (unk);
    \end{tikzpicture}
  \end{center}
\end{frame}

\begin{frame}{Two traditions}
  \begin{block}{'Unification style'}
    \textbf{Finite} set of markers denoting 'holes' ($\langle s \rangle, \langle o \rangle, \langle \textsf{mod} \rangle, \langle \textsf{comp} \rangle$) waiting to be filled with semantic values. Markers accessible in unconstrained order [Copestake \& all, 2001, Courcelles \& Englefriet 2012, Groshwitz \& all 2017]. 
  \end{block}
  \begin{itemize}
  \item<2>[\alert{$\rightarrow$}] number of 'holes' accessible at a given time of the construction process is bounded: '\alert{bounded memory}'.
  \end{itemize}
  \begin{block}{'Lambda style'}
    \textbf{Countably infinite} ordered set of markers but order constrain access (variables' scope) [Montague 1977, Steedman 2001].
  \end{block}
\end{frame}

\begin{frame}{Question}
  \begin{block}{'bounded memory' operators are popular}
  \begin{itemize}
  \item In semantic parsing [Chiang \& all 2013, Groschwitz \& all 2018, Chen \& all 2018].
  \item For the manual design of grammars [Bender 2002 \emph{inter alia}].
  \end{itemize}
  \end{block}

  \begin{alertblock}{Expressive limitation due to bounded memory capacity?}
    \begin{itemize}
    \item Specifically, considering long distance dependencies.
    \item If impossible (from distance) to combine a predicate with its argument right away $\rightarrow$ need to store argument slot until argument becomes available.
    %\item Link between compositionality, projectivity, and memory capacity?
    \end{itemize}
  \end{alertblock}
\end{frame}

\begin{frame}{Further motivation}
  
    \begin{itemize}
    \item A lot is known on expressive capacity of grammatical formalisms -- in terms of languages (of words/trees).
      \begin{itemize}
      \item \emph{e.g.}, famous CCG/TAG/LIG [Vijay-Shanker \& Weir, 1994] weak equivalence result.
      \end{itemize}
    \item What about the joint expressivity of grammatical formalisms and specific semantic combinators in terms of \emph{relations}?
    \item \alert{Do (weakly) equivalent grammatical formalisms support the same compositional interpretations?}
    \item Inform the elaboration of semantic parsing systems
    \end{itemize} 
  %\begin{block}{Inform the elaboration of semantic parsing systems}
  %  In particular systems relying on grammatical assumptions such as an adjacency principle (transition-based parsers, symbolic or neuronal). 
  %\end{block}
\end{frame}

\hidden{
\begin{block}{Questions}
    \begin{itemize}
    \item Quel type de syntaxe pour quel type de s\'emantique?
    \item Est-ce que des formalismes syntaxiques qui d\'ecrivent les m\^emes langages supportent les m\^emes interpr\'etations compositionnelles?
      %\item Important d'un point de vue th\'eorique mais aussi pour la conception de syst\`emes compositionnels de parsing s\'emantique.
    \end{itemize}
  \end{block}
\end{frame}
}

\hidden{
\begin{frame}{Main contribution}
  %\begin{block}{In a nutshell}
    \textbf{Compositional mechanisms of semantic interpretation which are both projective and with bounded memory are strictly less expressive than compositional non-projective ones with bounded memory.}
  %\end{block}

  %\begin{block}{Projectivity (roughly)}
  %  Every node of the syntax tree concatenate the sub-expressions generated by its children. (\emph{e.g.} S['Morgane wants to sleep'] $\rightarrow$ NP['Morgane'] $+$ VP['wants to sleep'] ).
  %\end{block}
  
\end{frame}
}

\hidden{
\begin{frame}{M\'emoire: syntaxe \emph{vs} s\'emantique}
  Les m\'echanismes compositionels projectifs \`a m\'emoire finie sont strictement moins expressifs que les m\'echanismes non-projectifs \`a m\'emoire finie.

  \begin{block}{Remarque}
    \begin{itemize}
    \item Projectivit\'e: aussi une notion syntactique de m\'emoire.
    \item Chaque constituant syntactique produit un fragment de l'expression finale repr\'esentable par 2 indices.
    \item ${}_0\textnormal{Morgane}_1 \textnormal{veut}_2 \textnormal{dormir}_3$: $NP_{[0,1]}$, $VP_{[1,3]}$, $S_{[0,3]}$.
    \item \textbf{impossible d'\'echanger une quantit\'e finie de m\'emoire `syntactique' contre une quantit\'e finie de m\'emoire `s\'emantique'}. 
    \end{itemize}
  \end{block}
   
\end{frame}
}

\hidden{
\begin{frame}{Plan de l'expos\'e}
  \begin{enumerate}
  \item Concepts.
  \item Intuition: projectivit\'e et capacit\'e m\'emorielle.
  \item R\'esultats.
  \item Discussion.
  \end{enumerate}
\end{frame}

\begin{frame}{Concepts}
  \begin{enumerate}
  \item \textbf{Grammaires}. %et formalismes grammaticaux..}
  \item M\'echanisme compositionnel.
  \item Capacit\'e m\'emorielle.
  %\item Classes de relation.
  \end{enumerate}
\end{frame}
}

\begin{frame}{Abstract view on grammars}
  \begin{itemize}
  \item Set of `grammatical' syntax trees $\{t_1, t_2, \dots \}$.
  \item \emph{yield} function, $\mathsf{yd}$, associating each tree with its string projection (the linguistic expression for which it is a grammatical analysis).
  \end{itemize}

  \begin{center}
    \begin{tikzpicture}
      \node[yshift=10pt] (t1) at (0,0) {$t_1$};
      \node (a1) at (0,-1) {};
      \path[draw] (0,0) -- (-0.5,-1) -- (0.5,-1)--cycle;
     
      \begin{scope}[xshift=1.5cm]
        \node[yshift=10pt] (t2) at (0,0) {$t_2$};
        \node (a2) at (0,-1) {};
        \path[draw] (0,0) -- (-0.5,-1) -- (0.5,-1)--cycle;
      \end{scope}
      
      \node[xshift=2cm] at (0.5, -0.5) {$\dots$};
      \node[xshift=2cm, yshift=-1.5cm] at (0.5, -0.5) {$\dots$};
      
      \begin{scope}[xshift=3.5cm]
        \node[yshift=10pt] (tk) at (0,0) {$t_k$};
        \node (ak) at (0,-1) {};
        \path[draw] (0,0) -- (-0.5,-1) -- (0.5,-1)--cycle;
      \end{scope}
      \node[xshift=4cm] at (0.5, -0.5) {$\dots$};
      \node[xshift=4cm, yshift=-1.5cm] at (0.5, -0.5) {$\dots$};
      
      \node (exp1) at (1,-2) {$\mathsf{exp}_1$};
      \node (expl) at (3.5,-2) {$\mathsf{exp}_l$};

      \path (a1) edge[decorate, decoration=snake, thick, ->] node[below left, midway] {$\mathsf{yd}$} (exp1);
      \path (a2) edge[decorate, decoration=snake, thick, ->] (exp1);
      \path (ak) edge[decorate, decoration=snake, thick, ->] (expl);
      
    \end{tikzpicture}
  \end{center}
  \begin{itemize}
  \item The set $\{t_1, t_2, \dots \}$ could be given by any kind of descriptive/computing device (formal grammar, neural net,\dots).
  \end{itemize}
\end{frame}

\begin{frame}{The projective yield $\mathsf{yd}_0$}
  \begin{itemize}
  \item Concatenates children's yield from left to right.
  \end{itemize}

  \begin{center}
    \scalebox{0.8}{
      \begin{forest}
        [S, name=snode
          [NP, name=npnode
            [N
              [Morgane]
            ]
          ]
          [VP, name=vpnode
            [V
              [wants]
            ]
            [VCOMP
              [TO
                [to]
              ] 
              [VB
                [sleep]
              ] 
            ]
          ]
        ]
        \node[xshift=2cm, draw, rectangle, blue] (s1) at (current bounding box.east) {Morgane}; \node[draw,rectangle, blue, right= 10pt of s1, yshift=1pt] (s2)  {wants to sleep};
        \node[xshift=-5pt, yshift=-5pt] (r1) at (s1.south west) {};
        \node[xshift=5pt, yshift=5pt] (r2) at (s2.north east) {};
        \draw (r1) rectangle (r2);
        \path(snode) edge[decorate, decoration=snake, ->, bend left, thick] node[midway, above]{$\mathsf{yd}_0$} (r2);
        \path (npnode) edge[blue, decorate, decoration=snake, ->, bend right, thick]  (s1);
        \path (vpnode) edge[blue, decorate, decoration=snake, ->, bend left, thick] (s2);
      \end{forest}
      }
  \end{center}
  
  
\end{frame}

\begin{frame}{A non-projective yield: $\mathsf{yd}_w$}
  \begin{block}{Swiss-German cross-serial dependencies [Shieber 1985]}
    \begin{center}
      \scalebox{0.8}{
        \begin{forest}
          [
            $\alpha_{\textnormal{l\"ond}}$
            [$\alpha_{\textnormal{d'chind}}$]
            [$\beta_{\textnormal{h\"alfed}}$, name=nodehelp
              [$\alpha_{\textnormal{Hans}}$]
              [$\beta_{aastriche}$
                [$\alpha_{huus}$]
              ]
            ]
          ]      
          \node[xshift=0.5cm, inner sep=0, outer sep=0] (s1) at (current bounding box.east) {d'chind};
          \node[right=5pt of s1, outer sep=0, draw, rectangle, blue] (s2) {em Hans es huss};
          \node[right=5pt of s2, inner sep=0, outer sep=0] (s3)  {l\"ond};
          \node[right=5pt of s3, outer sep=0, draw, rectangle, blue] (s4)  {h\"alfed aastriche};
        \path (nodehelp) edge [decorate, decoration=snake, ->, thick, blue, bend left] (s2);
        \path (nodehelp) edge [decorate, decoration=snake, ->, thick, blue, bend left] node[xshift=-0.5cm, very near start, above] {$\mathsf{yd}_w$} (s4);
        \end{forest}
      }
    \end{center}
    
    {\small
    \begin{exe}
      \sn
      \gll (dass) (mer) {d' chind} {em Hans} {es huus} lönd hälfed aastriiche\\
       (that) (we)  {the-children-ACC} {Hans-DAT} {the-house-ACC} let help
      paint\\
      \glt `(that we) let the children help Hans paint the house'
    \end{exe}
   }
  \end{block}
\end{frame}

\hidden{
\begin{frame}{Formalisme grammatical:}
  \begin{itemize}
    \item Collection quelconque de grammaires. 
  \end{itemize}
  \begin{exampleblock}{Exemples}
    \begin{itemize}
    \item[F_0] (Arbres de d\'erivations des grammaires hors-contextes, $\mathsf{red}_0$) 
    \item[F_{TAG}] (Arbres de d\'erivation des grammaires TAG, $\mathsf{red}_{w}$).
    \item[F_{dTAG}] (Arbres d\'eriv\'es des grammaires TAG, $\mathsf{red}_0$).
    \end{itemize}
    2 \& 3  d\'ecrivent les m\^emes langages (de mots), mais pas les m\^emes arbres. 
  \end{exampleblock}
\end{frame}
}

\hidden{
\begin{frame}{Concepts}
  \begin{enumerate}
  \item Grammaires. %et formalismes grammaticaux..
  \item \textbf{M\'echanismes compositionnels.}
  \item Capacit\'e m\'emorielle.
  %\item classes de relations.
  \end{enumerate}
\end{frame}
}
\begin{frame}{Semantic composition 1/3}
  \begin{block}{Interpretation for elementary syntactic constituants}
  
    \begin{center}
    \begin{tikzpicture}[intp/.style = {double, semithick, ->}]
      \node (tm) at (-3.5,0) {\small Morgane}; 
      \node (tv) at (0,0) {\small wants};
      \node (td) at (3.5,0) {\small sleep};
      
      \begin{scope}[yshift=-1cm]
        \node (gm) at (-3.5,0) {$\langle r \rangle$};
        \path (gm) edge[->, loop left] node[left] {\small Morgane} (gm);
        
        \node (gv) at (0,0) {$\langle r \rangle$};
        \path (gv) edge[->, loop left] node[left] {\small want} (gv);
        \node (sgv) at (-0.75,-1) {\small $\langle$s$\rangle$};
        \node (ogv) at (0.75,-1) {\small $\langle$o$\rangle$};
        
        \node (gd) at (3.5,0) {$\langle r \rangle$};
        \path (gd) edge[->, loop right] node[right] {\small sleep} (gd);
        \node (sgd) at (3.5,-1) {\small $\langle$s$\rangle$};

        \path (gv) edge[->] node [midway, left] {\small $a_0$} (sgv);
        \path (gv) edge[->] node [midway, right] {\small $a_1$} (ogv);

        \path (gd) edge[->] node [midway, left] {\small $a_0$} (sgd);
      \end{scope}
      \path (tm) edge[intp] (gm);
      \path (tv) edge[intp] (gv);
      \path (td) edge[intp] (gd);      
    \end{tikzpicture}
    \end{center}
  \end{block}
  \begin{itemize}
  \item $\langle s \rangle$, $\langle o \rangle$, $\langle r \rangle$: markers.
  \item $\langle s \rangle$, $\langle o \rangle$: argument placeholders ('holes'): a semantic value will eventually be substituted for them during the process of semantic composition.
  \item $\langle r \rangle$: root of the semantic constituant ('hook'), destined to be substituted for an argument placeholder.
  \end{itemize}
\end{frame}

\begin{frame}{Semantic composition 2/3}
  \begin{block}{Semantic algebra (\emph{i.e.} composition operators)}
    \begin{itemize}
    \item Example with the AM algebra [Groschwitz \& all 2017]
    \end{itemize}
    %\begin{exampleblock}{Ex: AM algebra [Groschwitz \& all 2017]}
    \begin{minipage}{0.65\linewidth}
      \scalebox{0.9}{
        \begin{tikzpicture}
          \node (gv) at (0,0) {$\langle r \rangle$};
          \path (gv) edge[->, loop left] node[left] (plh) {\small want} (gv);
          \node (sgv) at (-0.75,-1) {\small $\langle s\rangle$};
          \node (ogv) at (0.75,-1) {\small $\langle o \rangle$};
          
          \node (gd) at (2.5,0) {$\langle r \rangle$};
          \path (gd) edge[->, loop right] node[right] (prh) {\small sleep} (gd);
          \node (sgd) at (2.5,-1) {\small $\langle s \rangle$};
          
          \path (gv) edge[->] node [midway, left] {\small $a_0$} (sgv);
          \path (gv) edge[->] node [midway, right] (comp) {\small $a_1$} (ogv);
      
          \path (gd) edge[->] node [midway, left] (subj) {\small $a_0$} (sgd);
          \path (comp) -- node[midway] (m) {} (subj);
          \node[yshift=0.5cm] (m') at (m) {};
          \node[yshift=-0.5cm] (m'') at (m) {};
          \draw (m') -- (m'');
          
          \node[yshift=-0.5cm] (c1) at (m') {};
          \node[yshift=0.5cm] (c2) at (m) {};

          %\node[yshift = -1.25cm, xshift=-1.25cm] (c3) at (m') {};
          %\node[yshift = -1.25cm, xshift=1.25cm] (c4) at (m') {};

          \draw[draw, dashed, blue, thick, <->] (ogv) ..controls (c1) and (c2) .. (gd);
          \path (sgv) edge[draw, dashed, blue, thick, <->, bend right] (sgd);
          
          \draw[red,dashed]  (gd.north west) -- (gd.south east);
          \draw[red,dashed]  (gd.north east) -- (gd.south west);

          \draw[red,dashed]  (ogv.north west) -- (ogv.south east);
          \draw[red,dashed]  (ogv.north east) -- (ogv.south west);

          \node[yshift = -2cm] (plb) at (plh.north west) {};
          \path (plh.north west) edge[bend right=10] node[midway, left] {$App_o$} (plb);
          \node[yshift = -2cm] (prb) at (prh.north east) {};
          \path (prh.north east) edge[bend left=10] node[midway, right] {$=$} (prb);
        \end{tikzpicture}
      }
    \end{minipage}
    \begin{minipage}{0.34\linewidth}
      \scalebox{0.9}{
        \begin{tikzpicture}  \node (gv) at (0,0) {$\langle r \rangle$};
          \path (gv) edge[->, loop left] node[left] {\small want} (gv);
          \node (sgv) at (-0.75,-1) {\small $\langle s \rangle$};
          \node (gd) at (0.75,-1) {};
          \path (gv) edge[->] node [midway, left] {\small $a_0$} (sgv);
          \path (gv) edge[->] node [midway, right] {\small $a_1$} (gd);
          \path (gd) edge[->, loop below] node[right] {\small sleep} (gd);
          \path (gd) edge[->] node [midway, above] (subj) {\small $a_0$} (sgv);      
        \end{tikzpicture}
      }
    \end{minipage}
    
    %\end{exampleblock}
    \begin{itemize}
    \item Merge referenced marker $\langle o \rangle$ of the fonctor with the root $\langle r \rangle$ of the argument, then `forgets' these two markers.
    \item Merge any other identical marker (here, $\langle s \rangle$).
    \end{itemize}
  \end{block}
\end{frame}

\begin{frame}{Semantic composition 3/3}
  \begin{block}{Homomorphic interpretation of syntax trees}
    $\{ VP(x_1, x_2) \rightarrow APP_o(x_1, x_2),  S(x_1, x_2) \rightarrow APP_s(x_2, x_1) \}$
    \begin{center}
      \scalebox{0.8}{
      \begin{forest}
        [S, name = root, tikz = {
            \begin{scope}[xshift = -5.5cm, yshift=-1cm]
              \node  (v) at (1.25,1.25) {$\langle r \rangle$};
              \path (v) edge[loop left] node[left] {\small $\mathsf{want}$} (v);
              \node  (m) at (0.5,0) {};
              \path (m) edge[loop left] node[above] {\small $\mathsf{Morgane}$} (m);
              \node (d) at (2,0) {};
              \path (d) edge[loop right] node[above] {\small $\mathsf{sleep}$} (d);
              \path (v) edge[->] node[left, midway] (ref) {\small $\mathsf{a}_0$} (m);
              \path (v) edge[->] node[right, midway] {\small $\mathsf{a}_1$} (d);
              \path (d) edge[->] node[above] (a0) {\small $\mathsf{a}_0$} (m);
              \path (root) edge[double, ->] node[midway, above]{\small $App_s(x_2, x_1)$} (v); 
            \end{scope}
          }
          [NP, name = npnode, tikz = {\node[xshift=-1.5cm, yshift=-1.5cm] (gm) at (npnode) {$\langle r \rangle$};
              \path (gm) edge[->, loop left] node[left] {\small Morgane} (gm);
              \path (npnode) edge[double, ->, bend right] node[midway, above]{\small $x_1$} (gm);
          }
          [N
            [Morgane]
          ]
        ]
        [VP, name=vpnode, tikz={
            \begin{scope}[xshift = 4cm, yshift=-1.5cm]
              \node (gv) at (0,0) {$\langle r \rangle$};
              \path (gv) edge[->, loop left] node[left] {\small want} (gv);
              \node (sgv) at (-0.75,-1) {\small $\langle$s$\rangle$};
              \node (gd) at (0.75,-1) {};
              \path (gv) edge[->] node [midway, left] {\small $a_0$} (sgv);
              \path (gv) edge[->] node [midway, right] {\small $a_1$} (gd);
              \path (gd) edge[->, loop below] node[right] {\small sleep} (gd);
              \path (gd) edge[->] node [midway, above] (subj) {\small $a_0$} (sgv);
            \end{scope}
            \path (vpnode) edge[bend left, double, ->] node [midway, above]{\small $x_2$} (gv);
          }
          [V
            [wants]
          ]
          [VCOMP
            [TO, l=2mm [to, l=2mm]]
            [VB, l=2mm
              [sleep, l=2mm]
            ]           
          ]
        ]
        ]
      \end{forest}
      }
    \end{center}
  \end{block}
  
\end{frame}

\hidden{
\begin{frame}
  \begin{enumerate}
  \item Grammaires. %et formalismes grammaticaux..
  \item M\'echanismes compositionnels.
  \item \textbf{Capacit\'e m\'emorielle.}
  %\item classes de relations.
  \end{enumerate}
\end{frame}
}

\begin{frame}{'Semantic' memory}
  
    \begin{center}
      \begin{forest}
        [{S$\langle r \rangle$}
          [NP
            [N
              [{Morgane$\langle r \rangle$}]
            ]
          ]
          [{VP$\langle r, s \rangle$} 
            [V
              [{\bf \color{blue}wants$\langle r, s, o \rangle$}]
            ]
            [VCOMP
              [TO [{to$\langle \rangle$}]]
              [VINF
               [{sleep$\langle r, s \rangle$}]
             ]
           ]
          ]
        ]
        \node[xshift=1cm, yshift=1cm] at (current bounding box.east) {\color{blue} $3$ markers required};  
      \end{forest}
    \end{center}
 
\end{frame}

\hidden{
\begin{frame}
  \begin{enumerate} 
  \item Grammaires. %et formalismes grammaticaux..
  \item M\'echanismes compositionnels.
  \item Capacit\'e m\'emorielle.
  %\item \textbf{Classes de relations.}
  \end{enumerate}
\end{frame}


\begin{frame}{Classes de relations}
  \begin{itemize}
  \item F formalisme grammatical.
  \item $n$ capacit\'e m\'emorielle max.
  \end{itemize}
  \[ \mathcal{R}(F, n)\]
  relations expressions linguistiques/repr\'esentations s\'emantiques expressibles compositionnellement avec une grammaire de $F$ et une capacit\'e m\'emorielle $\le n$.
\end{frame}
}

\begin{frame}{Projectivity and memory 1/3}
  \begin{center}
    \begin{tikzpicture}
      \node (s) at (0,0) {d'chind em Hans es huss {\color{blue} l\"ond} {\color{darkgreen} h\"alfed} {\color{red} aastriche}};

      \begin{scope}[yshift=4cm]
      \node (g) at (0,0)  {$\langle r \rangle$};
      \path (g) edge[loop left, ->] node[left]{\color{blue} let} (g);

      \node (e) at (0,-1) {.};
      \path (e) edge[loop left, ->] node[left]{children} (e);

      \node (f) at (1,0)  {.};
      \path (f) edge[loop right, ->] node[right]{\color{darkgreen} help} (f);

      \node (h) at (1.5,-1)  {.};
      \path (h) edge[loop right, ->] node[right]{\color{red} paint} (h);

      \node (i) at (1,-1)  {.};
      \path (i) edge[loop below, ->] node[below]{Hans} (i);

      \node (j) at (2,-2)  {.};
      \path (j) edge[loop below, ->] node[below]{house} (j);

      \path (g) edge[->] node[midway, above] {$a_2$} (f);
      \path (g) edge[->] node[midway, left] {$a_1$} (e);

      \path (f) edge[->] node[midway, right] {$a_2$} (h);
      \path (f) edge[->] node[midway, left] {$a_1$} (i);

      \path (h) edge[->] node[midway, right] {$a_1$} (j);
           
      \end{scope}
    \end{tikzpicture}
  \end{center}

  \begin{itemize}
  \item {\color{blue} l\"ond}: $\langle r, o_1, o_2 \rangle$ (two objects).
  \item {\color{darkgreen} h\"alfed}: $\langle r, o_1, o_2 \rangle$ (two objects).
  \item {\color{red} aastriche}: $\langle r, o_1 \rangle$ (one object).
  \end{itemize}
\end{frame}

\begin{frame}{Projectivity and memory 2/3}
  Non-projective analysis possible with a $3$-markers capacity.
  \begin{center}
    \begin{forest}
      [{$\alpha_{\textnormal{l\"ond}} \langle r, {\color{orange} \cancel{o_1}}, {\color{purple} \cancel{o_2}} \rangle$}
        [{$\alpha_{\textnormal{d'chind}} \langle {\color{orange} r} \rangle$}]
        [{$\beta_{\textnormal{h\"alfed}} \langle {\color{purple}r}, {\color{red} \cancel {o'_1}}, {\color{darkgreen} \cancel{ o'_2} } \rangle$}
          [{$\alpha_{\textnormal{Hans}} \langle {\color{red} r} \rangle$}]
          [{$\beta_{aastriche}\langle {\color{darkgreen}r}, {\color{blue} \cancel{o'_1}} \rangle$}
            [{$\alpha_{huus} \langle {\color{blue} r} \rangle$}]
          ]
        ]
      ]
    \end{forest}
  \end{center}
\end{frame}


\begin{frame}{Projectivit\'e and memory 3/3}
  With a projective analysis: 4 markers seem intuitively required.
  \begin{center}
    \scalebox{0.7}{
      \begin{forest}
          [
            S
            [NP
              [d'chind]          
            ]
            [VP
              [NP
              [em Hans]
              ]
              [
                [VP
                  [NP
                    [es huus]
                  ]
                  [{\textbf{VP$\langle r, o_1, o'_1, o''_1 \rangle$}}
                    [{VP$\langle r, o_1, o'_1, {\color{darkgreen} o'_2} \rangle$}
                      [VP
                        [{l\"ond$\langle r, o_1, {\color{blue} o_2} \rangle$}]
                      ]
                      [{h\"alfed$\langle {\color{blue} r}, o'_1, {\color{darkgreen} o'_2} \rangle$}]
                    ]
                    [{aastriche$\langle {\color{darkgreen}r}, o''_1 \rangle$}]
                  ]
                ]
              ]
            ]
          ]
        \end{forest}
    }
    \end{center}
\end{frame}

\begin{frame}{Abstracting away}
  \begin{itemize}
  \item Arbitrary long crossed-serial dependencies $\rightarrow$ infinite memory required?
  \item A formal relation for a mathematical proof:
  \end{itemize}

  \begin{block}{$\mathsf{CSD}$}
    Word to graph function $w \mapsto g_w$ where
    \begin{itemize}   
    \item Words $w$ are of the form: $\underbrace{a \dots a}_{\textnormal{n times}}\underbrace{b \dots b}_{\textnormal{m times}}\underbrace{c \dots c}_{\textnormal{n times}}\underbrace{d \dots d}_{\textnormal{m times}}$.
    \item And for each such $w$, $g_w$ is:
      \begin{tikzpicture}
        \node[draw, circle] (a) at (0,0) {};
        \node[draw, circle] (c) at (0,-1) {}; 

        \node (do1) at (1,0) {\dots};

        \node[draw, circle] (an) at (2,0) {};
        \node[draw, circle] (cn) at (2,-1) {};
        
        \node[draw, circle] (b) at (3,0) {};
        \node[draw, circle] (d) at (3,-1) {};

        \node[] (do2) at (4,0) {\dots};
        \node[draw, circle] (bn) at (5,0) {};
        \node[draw, circle] (dn) at (5,-1) {};

        \node[draw, circle] (last) at (6,0) {};


        \path (a) edge[->] node[midway, left] {$c$} (c);
        \path (a) edge[->] node[midway, above] {$a$} (do1);
        \path (do1) edge[->] node[midway, above] {$a$} (an);
        \path (an) edge[->] node[midway, left] {$c$} (cn);
        \path (an) edge[->] node[midway, above] {$a$} (b);
        \path (b) edge[->] node[midway, left] {$d$} (d);
        \path (b) edge[->] node[midway, above] {$b$} (do2);
        \path (do2) edge[->] node[midway, above] {$b$} (bn);
        \path (bn) edge[->] node[midway, left] {$d$} (dn);
        \path (bn) edge[->] node[midway, above] {$b$} (last);
        
      \end{tikzpicture}
    \end{itemize}
  \end{block}
  
\end{frame}

\begin{frame}{Unnatural constructions}
  \begin{alertblock}{\only<2->{\textbf{NOT A}} Theorem \only<1>{?}}
    There exists no projective grammar and finite memory compositional interpretation mechanism over a projective grammar which expresses $\mathsf{CSD}$.
  \end{alertblock}

  \uncover<3>{
    \begin{center}
      \includegraphics[width=0.8\textwidth]{pics/pic-stupid-derivation.pdf}
    \end{center}
  }
\end{frame}


\begin{frame}{Theorem}
  If one further impose specific alignements between elementary syntactic and semantic constituants ('$a$' aligned with '\begin{tikzpicture}[inner sep=0pt] \node[] (l) {$\cdot$}; \node[xshift=20pt] at (l) (l') {$\cdot$};\path(l) edge[->] node[above]{a} (l'); \end{tikzpicture}', '$b$' aligned with '\begin{tikzpicture}[inner sep=0pt] \node[] (l) {$\cdot$}; \node[xshift=20pt] at (l) (l') {$\cdot$};\path(l) edge[->] node[above]{b} (l'); \end{tikzpicture}' \dots) %'Hans' with \begin{tikzpicture}[inner sep=0pt] \node[draw, circle] (l) {\phantom{x}};\path(l) edge[loop right] node[right]{Hans} (l); \end{tikzpicture}, \dots)
  it can be shown:

  \begin{block}{Theorem}
    \begin{itemize}
    \item \textbf{there exists no projective grammar and finite memory compositional interpretation mechanism over a projective grammar} expressing $\mathsf{CSD}$ and respecting elementary alignments.
    \item There exists a \textbf{non-projective} grammar and a finite-memory compositional interpretation mechanism expressing $\mathsf{CSD}$ and respecting elementary alignments.
    \item Remark: strong assumption on alignments but no assumption on grammatical formalism.
    \end{itemize}
  \end{block}
\end{frame}




\begin{frame}{Imperfect aligments}
  \begin{itemize}
  \item Whithout the alignment condition the theorem is false.
  \item However, weaker form of alignments can be achieved if we constrain the grammatical formalism (pumping lemma).
  \item \alert{Requires arbitrary complex 'arguments' to avoid previous unnatural constructions.} 
  \item Result for Tree-Adjoining Grammars (TAG).
  \end{itemize}

  \begin{block}{$\overline{\mathsf{CSD}}$ relation}
    \begin{minipage}{0.3\linewidth}
      \[a\textcolor{red}{\overline{aa}}b\textcolor{red}{\overline{b}}bc\textcolor{red}{\overline{c}}dd\]
    \end{minipage}
    \begin{minipage}{0.69\linewidth}
      \begin{center}
        \includegraphics[width=0.7\textwidth]{pics/pic-csd.pdf}
      \end{center}
    \end{minipage}
  \end{block}
  
\end{frame}

\begin{frame}{Two kinds of trees}
  \begin{itemize}
  \item TAG grammars produce derivation trees and derived trees.
  %\item Which one should we elect for semantic construction?
  %\item Deux `grammaires' distinctes dans notre d\'efinition.
  \end{itemize}

    \begin{minipage}{0.49\linewidth}
      \begin{center}
        \textbf{Derivation (dependency) tree}
        \begin{forest}
          [{$\alpha_{\textnormal{l\"ond}}$}, name=dep
            [{$\alpha_{\textnormal{d'chind}}$}]
            [{$\beta_{\textnormal{h\"alfed}}$}
              [{$\alpha_{\textnormal{Hans}}$}]
              [{$\beta_{aastriche}$}
                [{$\alpha_{huus}$}]
              ]
            ]
          ]
          %\node[above = of dep] {Arbre de d\'erivation (d\'ependences)};
        \end{forest}
      \end{center}
    \end{minipage}
    \begin{minipage}{0.5\linewidth}
      \begin{center}
        \textbf{Derived (syntagmatic) tree}
        \scalebox{0.7}{
          \begin{forest}
            [
              S, name=Snode
              [NP
                [d'chind]          
              ]
              [VP
                [NP
                  [em Hans]
                ]
                [
                  [VP
                    [NP
                      [es huus]
                    ]
                    [VP
                      [{VP}
                        [VP
                          [{l\"ond}]
                        ]
                        [{h\"alfed}]
                      ]
                      [{aastriche}]
                    ]
                  ]
                ]
              ]
            ]        
            %\node[above = of Snode] {Arbre d\'eriv\'e (syntagmatique)};
          \end{forest}
        }
      \end{center}
    \end{minipage}
\end{frame}

\begin{frame}{Two (weakly) equivalent grammar formalisms}
  \begin{itemize}
  \item Formalism $\mathsf{TAG}$: Use the {\bf derivation trees} of some TAG grammar with a \textbf{non-projective} yield.
  \item Formalism $\mathsf{PTAG}$: Use the {\bf derived trees} of some TAG grammar wutg the \textbf{projective} yield.
  \item The two formalisms generate the same \textbf{word langages}, but not necessarily the same \textbf{relations}.
  \end{itemize}

  \begin{center}
    \scalebox{0.5}{
      \begin{forest}
        [, phantom,
          [{$\alpha_{\textnormal{l\"ond}}$}
            [{$\alpha_{\textnormal{d'chind}}$}]
            [{$\beta_{\textnormal{h\"alfed}}$}
              [{$\alpha_{\textnormal{Hans}}$}]
              [{$\beta_{aastriche}$}
                [{$\alpha_{huus}$}]
              ]
            ]
          ]
          [
            S
            [NP
              [d'chind]          
            ]
            [VP
              [NP
                [em Hans]
              ]
              [
                [VP
                  [NP
                    [es huus]
                  ]
                  [VP
                    [{VP}
                      [VP
                        [{l\"ond}]
                      ]
                      [{h\"alfed}]
                    ]
                    [{aastriche}]
                  ]
                ]
              ]
            ]
          ]
        ]
      \end{forest}
    }
  \end{center}
\end{frame}



\begin{frame}{Second result}
  \begin{block}{Theorem}
    \begin{itemize}
    \item There exists a (non-projective) $\mathsf{TAG}$ grammar and a finite memory compositional interpretation mechanism expressing $\overline{\mathsf{CSD}}$.
    \item There exists no (projective) $\mathsf{PTAG}$ grammar and finite memory compositional interpretation mechanism expressing $\overline{\mathsf{CSD}}$.
    \end{itemize}
  \end{block}
\end{frame}

\hidden{
  \begin{frame}{Formalisation}
\end{frame}
}

\begin{frame}{Recap}
  \begin{itemize}
  \item Theoretical result on the link between compositionality,  projectivity and bounded memory capacity.
  \item Strong result, under strong assumption of perfect syntax/semantics aligmnents at the lexical level.
  \item \textbf{independent} of considered grammatical formalism.
    %\item D\'epend uniquement de la projectivit\'e de la grammaire. S'applique donc \`a des formalismes tr\`es expressifs, y compris mod\`eles neuronnaux proj(\emph{e.g.} la pluspart des syst\`emes shift/reduce).
  \item New light shed on the choice between derivation/derived tree as the support of semantic composition for TAG grammars.
  \item Do weakly equivalent grammatical formalisms support the same compositional interpretation mechanisms? $\rightarrow$ \textbf{No!}.
  \end{itemize}
\end{frame}

\begin{frame}{Conclusions and future work}
  \begin{itemize}
  \item Notion of expressivity at the syntax/semantics interface.
  \item Theoretical study on the link between projectivity and 'semantic' memory.
  \item What could we say about more restricted forms of non-projectivity? Finite increase in required memory capacity?
  \item Artificial non-projectivity due to imperfect aligners in semantic parsing systems.
  \item Locally translate from 'unification style' to 'lambda style' to circumvent projectivity issues?
  \end{itemize}
\end{frame}


\begin{frame}{}
  \begin{center}
    \huge Questions?
  \end{center}
\end{frame}

\hidden{
\begin{frame}{G\'en\'eralisation}
  \begin{itemize}
  \item D\'ependences crois\'ees arbitrairement longues.
  \item Une relation symbolique pour un r\'esultat formel:
  \end{itemize}

  \begin{block}{$\mathsf{CSD}$}
    \begin{itemize}
    \item Un langage symbolique et un graphe $g_w$ pour chaque mot $w$ du langage.
    \item Mots $w$ de la forme: $\underbrace{a \dots a}_{\textnormal{n fois}}\underbrace{b \dots b}_{\textnormal{m fois}}\underbrace{c \dots c}_{\textnormal{n fois}}\underbrace{d \dots d}_{\textnormal{m fois}}$.
    \item Interpr\'etation $g_w$ de la forme:
      \begin{tikzpicture}
        \node[draw, circle] (a) at (0,0) {};
        \node[draw, circle] (c) at (0,-1) {}; 

        \node (do1) at (1,0) {\dots};

        \node[draw, circle] (an) at (2,0) {};
        \node[draw, circle] (cn) at (2,-1) {};
        
        \node[draw, circle] (b) at (3,0) {};
        \node[draw, circle] (d) at (3,-1) {};

        \node[] (do2) at (4,0) {\dots};
        \node[draw, circle] (bn) at (5,0) {};
        \node[draw, circle] (dn) at (5,-1) {};

        \node[draw, circle] (last) at (6,0) {};


        \path (a) edge[->] node[midway, left] {$c$} (c);
        \path (a) edge[->] node[midway, above] {$a$} (do1);
        \path (do1) edge[->] node[midway, above] {$a$} (an);
        \path (an) edge[->] node[midway, left] {$c$} (cn);
        \path (an) edge[->] node[midway, above] {$a$} (b);
        \path (b) edge[->] node[midway, left] {$d$} (d);
        \path (b) edge[->] node[midway, above] {$b$} (do2);
        \path (do2) edge[->] node[midway, above] {$b$} (bn);
        \path (bn) edge[->] node[midway, left] {$d$} (dn);
        \path (bn) edge[->] node[midway, above] {$b$} (last);
        
      \end{tikzpicture}
    \end{itemize}
  \end{block}
  
\end{frame}


\begin{frame}{Th\'eor\`eme}
 
  \begin{itemize}
  \item $\mathsf{Tag}$: formalisme des grammaires d'arbres adjoints (non-projectif).
  \item $F$: formalisme grammatical projectif quelconque (= collection de grammaires projectives).
  \item $\mathcal{R}_{\leftrightarrow}(F, n)$: relations mots/graphes expressibles compositionnellement avec une grammaire quelconque de $F$, une capacit\'e m\'emorielle $\le n$, et en respectant les alignements entre lettres $x$ et arr\^etes $\begin{tikzpicture}[] \path (0,0) edge[->] node[above, midway] {$x$} (1,0); \end{tikzpicture}$. 
  \end{itemize}
  
  \begin{block}{Theor\`eme}
    \begin{itemize}
    \item $\mathsf{CSD} \in \mathcal{R}_{\leftrightarrow}(\mathsf{Tag}, 2)$.
    \item \textbf{Pour tout $F$ et $n$} $\mathsf{CSD} \notin \mathcal{R}_{\leftrightarrow}(F, n)$
    \end{itemize}
  \end{block}
\end{frame}



\hidden{
\begin{frame}{Diff\'erents types de marqueurs.}

  \begin{block}{Conscencus s\'emantique:}
    \begin{itemize}
    \item Remplacer un marqueur temporaire du foncteur par son argument.
    \end{itemize}
  \end{block}

  \begin{alertblock}{Divergence}
    \begin{itemize}
    \item Quel type de marqueur?
    \item Solution 1: nommer (par ex. $\langle s \rangle$) les marqueurs, et op\'erations de subsitution correspondantes ($App_c$) $\rightarrow$ marqueurs en nombre fini.
    \item Solution 2: 'structurer' les marqueurs en imposant un ordre d'utilisation $\rightarrow$ .
    \end{itemize} 
  \end{alertblock}

\end{frame}
}
}



\end{document}
